\section{Conclusions}


The objective of our model was to analyze the existing bus route system in Spartanburg, SC. Using an equation to quantify each service and amenity, we devised a value for each bus stop that would allow us to compare with alternative bus stops. 

Our model did not take into account any residential areas, which is a significant aspect of the model. The residential areas are where the general population lives and will commute to and from the services and amenities. The model did not consider "walk-ability" to and from the bus stops. Things that may decrease the "walk-ability" would be highways and poor sidewalks. The walking speed we devised by an average walking speed of a 65+ year old, which is not entirely inclusive of every passenger of the public transportation. Finally, we did not communicate with the general public of Spartanburg, SC. This would have brought us greater insight into the major necessities of the citizens of the city. 

The further work to be done to this model would be to collaborate with the community. Also, we would recommend compiling a value for each stop and analyzing a median value to create a threshold that would label each stop as "good" or "bad." This would finalize a solution that could be presented to Spartanburg County for further analysis and implementation. 
