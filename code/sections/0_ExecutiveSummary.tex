\section*{Executive Summary}

This paper is an analysis of the accessibility to services and amenities of Spartanburg's bus routes.  The city of Spartanburg, SC is growing and is home to many companies headquarters including Denny's and Milliken.  Since Spartanburg is a growing city, it is important to look at the public transportation and see how accessible it is for the average community member.  We examined a few bus stops on multiple bus routes in Spartanburg, SC and determined whether or not the bus stops were "good" based on their relative distance to different amenities.  Looking at existing maps of Spartanburg bus stops, we created a 0.6 mile radius around different stops and looked at what amenities were within the radius.  The 0.6 mile radius was used in assumption that the average person is willing to walk that far to reach any service or amenity.  

Another important assumption we had to make was the importance of our selected services and amenities.  We assumed importance based off of how many times someone would go to that particular amenity and then gave them a point value between 1 and 10.  For example, a grocery store is visited on average once a week, so it has the highest score of 10.  Once we had our point system organized, we created our model which is an equation that takes into consideration the assigned point values of different amenities, as well as the number of amenities that are in the 0.6 mile radius.  By solving the model, it gives a point value that can be used to determine if a bus stop is "good". 

One important limitation that we have in regard to this model is the lack of community input.  Based off of time constraints, there was not opportunity to interview community members to see how often people are using the bus system, where they are going, or where they need  to go. This model is based off of our assumptions of how amenities rank in importance, which is variable among people.  We also did not take into consideration how bus stops are located among different residential areas.

Using our full model analysis, we came to the conclusion that the Dorman Center Inbound Stop 2 is considered to be a good stop with an accessibility score of 74.  However, the South Liberty Inbound stop is considered to be a bad stop with an accessibility score of 36.  We observed multiple stops with varying different accessibility scores indicating that the bus routes in Spartanburg have different levels of accessibility.