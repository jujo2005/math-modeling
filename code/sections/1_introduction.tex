\section{Introduction}

\subsection{Problem Overview}

Spartanburg is a growing city in the Upstate region of South Carolina that is home to the headquarters of multiple reputable companies, such as Denny's and Milliken. As a city grows in population, its infrastructure needs to grow simultaneously in order to maintain a stable environment for both residents and industries. Public transportation is an aspect of city infrastructure that has become increasingly important in urban centers everywhere.  While Spartanburg has strong foundations in this area, improvements can be made. We were tasked with making improvements by determining a method to maximize accessibility to public services and amenities in Spartanburg through the means of the bus system. 

\subsection{Aims and Objectives}

As previously stated, the main objective of our model was to develop a method to maximize accessibility to services and amenities through the public bus system in Spartanburg, South Carolina. In order to do this, we sought to create something that can serve as a means of assessing accessibility for both existing stops and locations for potential new stops. Considering this, we came up with the following problem statement:
\linebreak
\linebreak
    \textit{Our model will predict the optimal places to put bus stops in Spartanburg by finding areas that would maximize the amount of reachable and important services and amenities.}

\subsection{Overview Statistics}

This is a problem with minimal existing research and data that are specific to the city of Spartanburg. However, we do have some insightful statistics on the infrastructure and residents of Spartanburg. After manually collecting data on all of the services and amenities within Spartanburg that fall into various categories (Appendix A), we can see that there are 16 pharmacies, 39 grocery stores, 15 laundromats, and 21 medical facilities available within the city. To put those numbers in context, the 2022 United States Census reports that there are 38,584 residents in the city of Spartanburg. This shows that there are plenty of valuable services and amenities available to the residents in Spartanburg that can accessed through the bus routes.
\linebreak
\linebreak
In addition to this data, we were also able to find data collected by the U.S. Census Bureau in 2022 that showed various statistics about employed workers in Spartanburg County (Appendix B). In this dataset, it shows the amount of vehicles that workers reported as being available to them, as well as showing the means of transportation that workers reported using in order to get to their jobs. After performing a simple data summary (Appendix C), we found that an estimated 20.8 percent of workers in the Spartanburg area have 1 or less cars available to them, which shows that a large portion of residents could realistically benefit from improved public transportation. Additionally, we found that the vast majority of workers that claimed to walk to work also reported having no vehicle available to them. This could indicate that the current bus system does not reach the employers of many residents in the area who are in critical need of efficient public transport due to a lack of personal vehicles. We feel that these statistics are good motivators for the problem we are seeking to solve, as they show the legitimate need from the community for an improved bus system.

